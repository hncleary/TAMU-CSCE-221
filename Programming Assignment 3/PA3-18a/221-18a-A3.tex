%% LyX 2.2.3 created this file.  For more info, see http://www.lyx.org/.
%% Do not edit unless you really know what you are doing.
\documentclass[english]{article}
\usepackage{mathptmx}
\usepackage{helvet}
\usepackage{courier}
\usepackage[T1]{fontenc}
\usepackage[latin9]{inputenc}
\usepackage{geometry}
\geometry{verbose,tmargin=1in,bmargin=1in,lmargin=1in,rmargin=1in,headheight=0in,headsep=0in}
\pagestyle{empty}
\usepackage{color}
\usepackage{babel}
\usepackage[unicode=true]
 {hyperref}

\makeatletter

%%%%%%%%%%%%%%%%%%%%%%%%%%%%%% LyX specific LaTeX commands.
%% Because html converters don't know tabularnewline
\providecommand{\tabularnewline}{\\}

%%%%%%%%%%%%%%%%%%%%%%%%%%%%%% Textclass specific LaTeX commands.
\newenvironment{lyxcode}
{\par\begin{list}{}{
\setlength{\rightmargin}{\leftmargin}
\setlength{\listparindent}{0pt}% needed for AMS classes
\raggedright
\setlength{\itemsep}{0pt}
\setlength{\parsep}{0pt}
\normalfont\ttfamily}%
 \item[]}
{\end{list}}

%%%%%%%%%%%%%%%%%%%%%%%%%%%%%% User specified LaTeX commands.
\date{}

\makeatother

\begin{document}
\begin{center}
\textbf{\large{}CSCE 221 Assignment 3 Cover Page}\\
\bigskip{}
\par\end{center}

First Name~~~~~~~~~~~~~~~~~~~~~~~~~~~~~~~~~~Last
Name ~~~~~~~~~~~~~~~~~~~~~~~~UIN~~~~~~~~~~~~~~\bigskip{}

User Name ~~~~~~~~~~~~~~~~~~~~~~~~~~~~~E-mail
address~~~~~~~~~~~~~~~~~~~~~~~~~~~~~~\medskip{}

Please list all sources in the table below including web pages which
you used to solve or implement the current homework. If you fail to
cite sources you can get a lower number of points or even zero, read
more on Aggie Honor System Office website: \texttt{\href{http://aggiehonor.tamu.edu/}{http://aggiehonor.tamu.edu/}}\medskip{}
\medskip{}
\noindent \begin{flushleft}
\begin{tabular}{|c|c|c|c|c|}
\hline 
Type of sources  & ~~~~~~~~~~~~~~~~~~~~~~~ & ~~~~~~~~~~~~~~~~~~~~~~~~ & ~~~~~~~~~~~~~~~~~~~~~~~ & ~~~~~~~~~~~~~~~~~~~~~~~\tabularnewline
 &  &  &  & \tabularnewline
\hline 
People &  &  &  & \tabularnewline
 &  &  &  & \tabularnewline
\hline 
Web pages (provide URL)  &  &  &  & \tabularnewline
 &  &  &  & \tabularnewline
\hline 
Printed material &  &  &  & \tabularnewline
 &  &  &  & \tabularnewline
\hline 
Other Sources  &  &  &  & \tabularnewline
 &  &  &  & \tabularnewline
\hline 
\end{tabular}
\par\end{flushleft}

\medskip{}
\medskip{}

\noindent I certify that I have listed all the sources that I used
to develop the solutions/codes to the submitted work.

\noindent \emph{On my honor as an Aggie, I have neither given nor
received any unauthorized help on this academic work}.

\bigskip{}
\bigskip{}

\begin{tabular}{cccccc}
Your Name  & ~~~~~~~~~~~~~~~~~~~~~~~~~~~ &  & ~~~~~~~~~~~~~~~~~~~~~ & Date  & ~~~~~~~~~~~~~~~~~~~~\tabularnewline
\end{tabular}

\title{\newpage{}}
\maketitle
\begin{center}
\textbf{\Large{}CSCE 221 Assignment 3 }
\par\end{center}{\Large \par}

\begin{center}
\textbf{\Large{}Spring 2018}
\par\end{center}{\Large \par}

\author{\emph{due to eCampus by March 9, and demonstration of Part 1 in labs
on February 26/27}}

\section*{Objective }

This is an individual assignment which has two parts. 
\begin{enumerate}
\item Part 1: C++ implementation of \texttt{DoublyLinkedList} for \texttt{int}
and generic types based on the provided supplementary code. 
\item Part 2: C++ implementation of \texttt{\textcolor{black}{MinQueue}}
data structure that can store \emph{comparable elements.} 
\end{enumerate}

\section*{Part 1: Implementation of DoublyLinkedList}
\begin{enumerate}
\item Untar supplementary code \texttt{221-A3-code.tar}. Use the 7-zip software
to extract the files in Windows, or use the following command in Linux.
\begin{lyxcode}
tar~xfv~221-A3-code.tar
\end{lyxcode}
\item \texttt{DoublyLinkedList} for integers
\begin{enumerate}
\item Most of the code is extracted from the lecture slides. An exception
structure is defined to complete the program.
\item You need to complete the functions which are declared in the header
file \texttt{DoublyLinkedList.h}.
\item Type the following commands to compile the program
\begin{lyxcode}
make
\end{lyxcode}
\item The main program includes examples of creating doubly linked lists,
and demonstrates how to use them. Type the following command to run
the executable file:
\begin{lyxcode}
./run-dll

\end{lyxcode}
\item Test the doubly linked list functions in \texttt{main}. 
\end{enumerate}
\item Implement a templated version of the class \texttt{DoublyLinkedList}
and test the functions for correctness. Follow the instructions below:
\begin{enumerate}
\item Templates should be declared and defined in a header \texttt{.h} file.
Move the content of \texttt{DoublyLinkedList.cpp} and \texttt{DoublyLinkedList.h}
to \texttt{TemplateDoublyLinkedList.h} 
\item Replace \texttt{int obj} by \texttt{T obj} in the \texttt{struct}
\texttt{DListNode} so the list nodes store generic \texttt{T} objects
instead of integers. Later on, when a \texttt{DListNode} object is
created, say, in the main function, \texttt{T} can be specified as
a \texttt{char}, \texttt{string} or a user-defined class.
\item To create a templated class with a generic type \texttt{T}, you must
replace a declaration/return type \texttt{int} by \texttt{T} (except
for the \texttt{count} variable). 
\begin{enumerate}
\item To use the generic type \texttt{T}, you must change each type declaration.
\item Use the generic type \texttt{T} anywhere throughout the class \texttt{TemplateLinkedList}.
\end{enumerate}
\item Add the keyword \texttt{template <typename T>} before a class declaration. 
\item In each member function signature, replace \texttt{DoublyLinkedList::}
by \texttt{DoublyLinkedList<T>::}
\item If a member function is defined outside the class declaration, change
the function signature, that is, replace \texttt{LinkedList::} by
\texttt{TemplateLinkedList<T>::}
\item To use the generic type \texttt{T} anywhere throughout the class \texttt{DListNode}
and \texttt{DoublyLinkedList}, you must declare (add) \texttt{template
<typename T>} before classes and member functions defined outside
the class declaration.
\end{enumerate}
\item Compile and run the generic version in a similar way as for \texttt{int}
type. Type the following commands to compile the program.
\begin{lyxcode}
make
\end{lyxcode}
\item The main program includes examples of creating doubly linked lists
of \texttt{string}, and demonstrates how to use them. Type the following
command to run the executable.
\begin{lyxcode}
./run-tdll
\end{lyxcode}
\end{enumerate}

\section*{Part 2: Implementation of MinQueue data structure based on DoublyLinkedList}

\textcolor{black}{The }\texttt{\textcolor{black}{MinQueue}} data structure
should store \emph{the comparable elements} that support the queue
operations: \texttt{\textcolor{black}{enqueue(x)}}\texttt{,} \texttt{\textcolor{black}{dequeue()}},
\texttt{\textcolor{black}{size()}},\texttt{\textcolor{black}{{} isEmpty()}},
and in addition the \texttt{\textcolor{black}{min()}}\textcolor{black}{{}
operation} that returns (but not deletes) the smallest value currently
stored in the queue. 

Use the adapter design pattern for implementation of \texttt{\textcolor{black}{MinQueue}}
that work together with the class \texttt{DoublyLinkedLists} defined
in the Part 1. The runtime worst case of all operations except \texttt{\textcolor{black}{min()}}
should be \textcolor{black}{\emph{constant,}} $O(1)$.

The implementation details of the \texttt{\textcolor{black}{MinQueue}}\textcolor{black}{{}
operations, justification of their running time, and tests for correctness
should be provided in the part 2 of the report. }

\section*{What to submit to eCampus?}
\begin{itemize}
\item Create a directory for the Part 1 that includes: \texttt{DoublyLinkedList}
source code for \texttt{int} and generic types, typed report with
description of the linked list implementation, complexity analysis
of code expressed in terms of big-O, and the test cases done for correctness.
\item Create a directory for the Part 2 that includes: \texttt{Min}Queue
source code, typed report with description of the \texttt{Min}Queue
class implementation, complexity analysis of code expressed in terms
of big-O, and the test cases done for correctness. 
\item Make a tar file that contains the Part 1 and Part 2 directories and
submit it to eCampus for grading.
\end{itemize}

\end{document}
